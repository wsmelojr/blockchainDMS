%% bare_jrnl.tex
%% V1.4b
%% 2015/08/26
%% by Michael Shell
%% see http://www.michaelshell.org/
%% for current contact information.
%%
%% This is a skeleton file demonstrating the use of IEEEtran.cls
%% (requires IEEEtran.cls version 1.8b or later) with an IEEE
%% journal paper.
%%
%% Support sites:
%% http://www.michaelshell.org/tex/ieeetran/
%% http://www.ctan.org/pkg/ieeetran
%% and
%% http://www.ieee.org/

%%*************************************************************************
%% Legal Notice:
%% This code is offered as-is without any warranty either expressed or
%% implied; without even the implied warranty of MERCHANTABILITY or
%% FITNESS FOR A PARTICULAR PURPOSE! 
%% User assumes all risk.
%% In no event shall the IEEE or any contributor to this code be liable for
%% any damages or losses, including, but not limited to, incidental,
%% consequential, or any other damages, resulting from the use or misuse
%% of any information contained here.
%%
%% All comments are the opinions of their respective authors and are not
%% necessarily endorsed by the IEEE.
%%
%% This work is distributed under the LaTeX Project Public License (LPPL)
%% ( http://www.latex-project.org/ ) version 1.3, and may be freely used,
%% distributed and modified. A copy of the LPPL, version 1.3, is included
%% in the base LaTeX documentation of all distributions of LaTeX released
%% 2003/12/01 or later.
%% Retain all contribution notices and credits.
%% ** Modified files should be clearly indicated as such, including  **
%% ** renaming them and changing author support contact information. **
%%*************************************************************************


% *** Authors should verify (and, if needed, correct) their LaTeX system  ***
% *** with the testflow diagnostic prior to trusting their LaTeX platform ***
% *** with production work. The IEEE's font choices and paper sizes can   ***
% *** trigger bugs that do not appear when using other class files.       ***                          ***
% The testflow support page is at:
% http://www.michaelshell.org/tex/testflow/



\documentclass[journal]{IEEEtran}
\usepackage[pdftex]{graphicx}
\usepackage{tabularx}
\usepackage{textcomp}
\usepackage{tablefootnote}
\usepackage{booktabs}
\usepackage{url}

% Some Computer Society conferences also require the compsoc mode option,
% but others use the standard conference format.
%
% If IEEEtran.cls has not been installed into the LaTeX system files,
% manually specify the path to it like:
% \documentclass[conference]{../sty/IEEEtran}





% Some very useful LaTeX packages include:
% (uncomment the ones you want to load)


% *** MISC UTILITY PACKAGES ***
%
%\usepackage{ifpdf}
% Heiko Oberdiek's ifpdf.sty is very useful if you need conditional
% compilation based on whether the output is pdf or dvi.
% usage:
% \ifpdf
%   % pdf code
% \else
%   % dvi code
% \fi
% The latest version of ifpdf.sty can be obtained from:
% http://www.ctan.org/pkg/ifpdf
% Also, note that IEEEtran.cls V1.7 and later provides a builtin
% \ifCLASSINFOpdf conditional that works the same way.
% When switching from latex to pdflatex and vice-versa, the compiler may
% have to be run twice to clear warning/error messages.


% *** CITATION PACKAGES ***
%
%\usepackage{cite}
% cite.sty was written by Donald Arseneau
% V1.6 and later of IEEEtran pre-defines the format of the cite.sty package
% \cite{} output to follow that of the IEEE. Loading the cite package will
% result in citation numbers being automatically sorted and properly
% "compressed/ranged". e.g., [1], [9], [2], [7], [5], [6] without using
% cite.sty will become [1], [2], [5]--[7], [9] using cite.sty. cite.sty's
% \cite will automatically add leading space, if needed. Use cite.sty's
% noadjust option (cite.sty V3.8 and later) if you want to turn this off
% such as if a citation ever needs to be enclosed in parenthesis.
% cite.sty is already installed on most LaTeX systems. Be sure and use
% version 5.0 (2009-03-20) and later if using hyperref.sty.
% The latest version can be obtained at:
% http://www.ctan.org/pkg/cite
% The documentation is contained in the cite.sty file itself.


% *** GRAPHICS RELATED PACKAGES ***
%
\ifCLASSINFOpdf
  % \usepackage[pdftex]{graphicx}
  % declare the path(s) where your graphic files are
  % \graphicspath{{../pdf/}{../jpeg/}}
  % and their extensions so you won't have to specify these with
  % every instance of \includegraphics
  % \DeclareGraphicsExtensions{.pdf,.jpeg,.png}
\else
  % or other class option (dvipsone, dvipdf, if not using dvips). graphicx
  % will default to the driver specified in the system graphics.cfg if no
  % driver is specified.
  % \usepackage[dvips]{graphicx}
  % declare the path(s) where your graphic files are
  % \graphicspath{{../eps/}}
  % and their extensions so you won't have to specify these with
  % every instance of \includegraphics
  % \DeclareGraphicsExtensions{.eps}
\fi
% graphicx was written by David Carlisle and Sebastian Rahtz. It is
% required if you want graphics, photos, etc. graphicx.sty is already
% installed on most LaTeX systems. The latest version and documentation
% can be obtained at: 
% http://www.ctan.org/pkg/graphicx
% Another good source of documentation is "Using Imported Graphics in
% LaTeX2e" by Keith Reckdahl which can be found at:
% http://www.ctan.org/pkg/epslatex
%
% latex, and pdflatex in dvi mode, support graphics in encapsulated
% postscript (.eps) format. pdflatex in pdf mode supports graphics
% in .pdf, .jpeg, .png and .mps (metapost) formats. Users should ensure
% that all non-photo figures use a vector format (.eps, .pdf, .mps) and
% not a bitmapped formats (.jpeg, .png). The IEEE frowns on bitmapped formats
% which can result in "jaggedy"/blurry rendering of lines and letters as
% well as large increases in file sizes.
%
% You can find documentation about the pdfTeX application at:
% http://www.tug.org/applications/pdftex





% *** MATH PACKAGES ***
%
%\usepackage{amsmath}
% A popular package from the American Mathematical Society that provides
% many useful and powerful commands for dealing with mathematics.
%
% Note that the amsmath package sets \interdisplaylinepenalty to 10000
% thus preventing page breaks from occurring within multiline equations. Use:
%\interdisplaylinepenalty=2500
% after loading amsmath to restore such page breaks as IEEEtran.cls normally
% does. amsmath.sty is already installed on most LaTeX systems. The latest
% version and documentation can be obtained at:
% http://www.ctan.org/pkg/amsmath





% *** SPECIALIZED LIST PACKAGES ***
%
%\usepackage{algorithmic}
% algorithmic.sty was written by Peter Williams and Rogerio Brito.
% This package provides an algorithmic environment fo describing algorithms.
% You can use the algorithmic environment in-text or within a figure
% environment to provide for a floating algorithm. Do NOT use the algorithm
% floating environment provided by algorithm.sty (by the same authors) or
% algorithm2e.sty (by Christophe Fiorio) as the IEEE does not use dedicated
% algorithm float types and packages that provide these will not provide
% correct IEEE style captions. The latest version and documentation of
% algorithmic.sty can be obtained at:
% http://www.ctan.org/pkg/algorithms
% Also of interest may be the (relatively newer and more customizable)
% algorithmicx.sty package by Szasz Janos:
% http://www.ctan.org/pkg/algorithmicx




% *** ALIGNMENT PACKAGES ***
%
%\usepackage{array}
% Frank Mittelbach's and David Carlisle's array.sty patches and improves
% the standard LaTeX2e array and tabular environments to provide better
% appearance and additional user controls. As the default LaTeX2e table
% generation code is lacking to the point of almost being broken with
% respect to the quality of the end results, all users are strongly
% advised to use an enhanced (at the very least that provided by array.sty)
% set of table tools. array.sty is already installed on most systems. The
% latest version and documentation can be obtained at:
% http://www.ctan.org/pkg/array


% IEEEtran contains the IEEEeqnarray family of commands that can be used to
% generate multiline equations as well as matrices, tables, etc., of high
% quality.




% *** SUBFIGURE PACKAGES ***
%\ifCLASSOPTIONcompsoc
%  \usepackage[caption=false,font=normalsize,labelfont=sf,textfont=sf]{subfig}
%\else
%  \usepackage[caption=false,font=footnotesize]{subfig}
%\fi
% subfig.sty, written by Steven Douglas Cochran, is the modern replacement
% for subfigure.sty, the latter of which is no longer maintained and is
% incompatible with some LaTeX packages including fixltx2e. However,
% subfig.sty requires and automatically loads Axel Sommerfeldt's caption.sty
% which will override IEEEtran.cls' handling of captions and this will result
% in non-IEEE style figure/table captions. To prevent this problem, be sure
% and invoke subfig.sty's "caption=false" package option (available since
% subfig.sty version 1.3, 2005/06/28) as this is will preserve IEEEtran.cls
% handling of captions.
% Note that the Computer Society format requires a larger sans serif font
% than the serif footnote size font used in traditional IEEE formatting
% and thus the need to invoke different subfig.sty package options depending
% on whether compsoc mode has been enabled.
%
% The latest version and documentation of subfig.sty can be obtained at:
% http://www.ctan.org/pkg/subfig




% *** FLOAT PACKAGES ***
%
%\usepackage{fixltx2e}
% fixltx2e, the successor to the earlier fix2col.sty, was written by
% Frank Mittelbach and David Carlisle. This package corrects a few problems
% in the LaTeX2e kernel, the most notable of which is that in current
% LaTeX2e releases, the ordering of single and double column floats is not
% guaranteed to be preserved. Thus, an unpatched LaTeX2e can allow a
% single column figure to be placed prior to an earlier double column
% figure.
% Be aware that LaTeX2e kernels dated 2015 and later have fixltx2e.sty's
% corrections already built into the system in which case a warning will
% be issued if an attempt is made to load fixltx2e.sty as it is no longer
% needed.
% The latest version and documentation can be found at:
% http://www.ctan.org/pkg/fixltx2e


%\usepackage{stfloats}
% stfloats.sty was written by Sigitas Tolusis. This package gives LaTeX2e
% the ability to do double column floats at the bottom of the page as well
% as the top. (e.g., "\begin{figure*}[!b]" is not normally possible in
% LaTeX2e). It also provides a command:
%\fnbelowfloat
% to enable the placement of footnotes below bottom floats (the standard
% LaTeX2e kernel puts them above bottom floats). This is an invasive package
% which rewrites many portions of the LaTeX2e float routines. It may not work
% with other packages that modify the LaTeX2e float routines. The latest
% version and documentation can be obtained at:
% http://www.ctan.org/pkg/stfloats
% Do not use the stfloats baselinefloat ability as the IEEE does not allow
% \baselineskip to stretch. Authors submitting work to the IEEE should note
% that the IEEE rarely uses double column equations and that authors should try
% to avoid such use. Do not be tempted to use the cuted.sty or midfloat.sty
% packages (also by Sigitas Tolusis) as the IEEE does not format its papers in
% such ways.
% Do not attempt to use stfloats with fixltx2e as they are incompatible.
% Instead, use Morten Hogholm'a dblfloatfix which combines the features
% of both fixltx2e and stfloats:
%
% \usepackage{dblfloatfix}
% The latest version can be found at:
% http://www.ctan.org/pkg/dblfloatfix


% *** PDF, URL AND HYPERLINK PACKAGES ***
%
%\usepackage{url}
% url.sty was written by Donald Arseneau. It provides better support for
% handling and breaking URLs. url.sty is already installed on most LaTeX
% systems. The latest version and documentation can be obtained at:
% http://www.ctan.org/pkg/url
% Basically, \url{my_url_here}.


% *** Do not adjust lengths that control margins, column widths, etc. ***
% *** Do not use packages that alter fonts (such as pslatex).         ***
% There should be no need to do such things with IEEEtran.cls V1.6 and later.
% (Unless specifically asked to do so by the journal or conference you plan
% to submit to, of course. )


% correct bad hyphenation here
\hyphenation{op-tical net-works semi-conduc-tor}


\begin{document}

%%%%%%%%%%%%%%%%%%%%%%%%%%%%%%%%%%%%%%%%%%%%%%%%%%%%%%%%
% REMOVENDO URL DAS REFERENCIAS BIBLIOGRAFICAS
% 1) Adicionar estrutura abaixo ao arquivo .bib
% @IEEEtranBSTCTL{MyBSTcontrol,
%     CTLuse_url = "no",
% }
% 2) Descomentar o comando abaixo:
%\bstctlcite{MyBSTcontrol} 
%%%%%%%%%%%%%%%%%%%%%%%%%%%%%%%%%%%%%%%%%%%%%%%%%%%%%%%%

% paper title
% Titles are generally capitalized except for words such as a, an, and, as,
% at, but, by, for, in, nor, of, on, or, the, to and up, which are usually
% not capitalized unless they are the first or last word of the title.
% Linebreaks \\ can be used within to get better formatting as desired.
% Do not put math or special symbols in the title.
%\title{How Blockchains can improve Measuring Instruments Regulation and Control}
\title{Using Blockchains to Implement Distributed Measuring Systems}

% author names and affiliations
% use a multiple column layout for up to three different
% affiliations
\author{\IEEEauthorblockN{Wilson Melo Jr, Luiz F. R. C. Carmo}
\IEEEauthorblockA{National Institute of Metrology,\\
Quality and Technology, RJ, Brazil\\
%\{wsjunior,lfrust\}@inmetro.gov.br
}
\and
\IEEEauthorblockN{Alysson Bessani, Nuno Neves}
\IEEEauthorblockA{LaSIGE, Faculdade de Ci\^encias\\
Universidade de Lisboa, Portugal\\
%\{bessani,nuno\}@fc.ul.pt
}
\and
\IEEEauthorblockN{Altair Santin}
\IEEEauthorblockA{Pontifical Catholic University of Parana\\
Curitiba, PR, Brazil\\
%santin@ppgia.pucpr.br
}
}

\author{Wilson~S.~Melo~Jr, %~\IEEEmembership{Member,~IEEE,}
        Alysson~Bessani, %~\IEEEmembership{Fellow,~OSA,}
        Nuno~Neves, %~\IEEEmembership{Life~Fellow,~IEEE}% <-this % stops a space
        Altair~Santin
        and Luiz~F.~R.~C.~Carmo
\thanks{W. Melo Jr. and Luiz F. R. C. Carmo are with the National Institute of Metrology, Quality and Technology, Duque de Caxias, RJ, Brazil; and with the Federal University of Rio de Janeiro, Brazil.}% <-this % stops a space
\thanks{A. Bessani and N. Neves are with LaSIGE, Faculdade de Ci\^encias, Universidade de Lisboa, Portugal.}% <-this % stops a space
\thanks{A. Santin is with Pontifical Catholic University of Parana, Curitiba, PR, Brazil.}% <-this % stops a space
%\thanks{Manuscript received July 28, 2018; revised Xxxx NN, NNNN.}
}

% conference papers do not typically use \thanks and this command
% is locked out in conference mode. If really needed, such as for
% the acknowledgment of grants, issue a \IEEEoverridecommandlockouts
% after \documentclass

% for over three affiliations, or if they all won't fit within the width
% of the page, use this alternative format:
% 
%\author{\IEEEauthorblockN{Michael Shell\IEEEauthorrefmark{1},
%Homer Simpson\IEEEauthorrefmark{2},
%James Kirk\IEEEauthorrefmark{3}, 
%Montgomery Scott\IEEEauthorrefmark{3} and
%Eldon Tyrell\IEEEauthorrefmark{4}}
%\IEEEauthorblockA{\IEEEauthorrefmark{1}School of Electrical and Computer Engineering\\
%Georgia Institute of Technology,
%Atlanta, Georgia 30332--0250\\ Email: see http://www.michaelshell.org/contact.html}
%\IEEEauthorblockA{\IEEEauthorrefmark{2}Twentieth Century Fox, Springfield, USA\\
%Email: homer@thesimpsons.com}
%\IEEEauthorblockA{\IEEEauthorrefmark{3}Starfleet Academy, San Francisco, California 96678-2391\\
%Telephone: (800) 555--1212, Fax: (888) 555--1212}
%\IEEEauthorblockA{\IEEEauthorrefmark{4}Tyrell Inc., 123 Replicant Street, Los Angeles, California 90210--4321}}


% use for special paper notices
%\IEEEspecialpapernotice{(Invited Paper)}

% The paper headers
%\markboth{IEEE Transactions on Instrumentation and Measurements, Special Issue I2MTC, July~2018}
\markboth{Using Blockchains to Implement Distributed Measuring Systems}{}
%{Melo Jr. \MakeLowercase{\textit{et al.}}: Using Blockchains to Implement Distributed Measuring Systems}
% The only time the second header will appear is for the odd numbered pages
% after the title page when using the twoside option.
% 
% *** Note that you probably will NOT want to include the author's ***
% *** name in the headers of peer review papers.                   ***
% You can use \ifCLASSOPTIONpeerreview for conditional compilation here if
% you desire.


% make the title area
\maketitle

% As a general rule, do not put math, special symbols or citations
% in the abstract
\begin{abstract}
In recent years, measuring instruments have become quite complex due to the integration of embedded systems and software components and the increasing aggregation of new features. 
Consequently, metrological regulation and control require more efforts from notified bodies, becoming slower and more expensive. 
In this work, we evaluate the use of blockchains as a resource to overcome such challenges.
%how blockchains can help to overcome such challenges. 
We start with a conceptual model for implementing measuring instruments in a distributed blockchain-based architecture, and compare it with traditional measuring instruments and distributed measuring models discussed in previous works. 
We also made a security analysis, demonstrating that blockchains-based measuring systems can impact the way measuring instruments are used in consumer relations while improving security and simplifying metrological regulation and control.
We implement a vehicle speed measuring system using the Hyperledger Fabric blockchain platform.
We evaluate the security and performance of our blockchain-based measuring system by executing tests with data from real speed meter sensors.
The results are promising and validate the feasibility of our idea.
Finally, we point out the main challenges related to our approach, suggesting alternatives and potential issues to be addressed by future works.
\end{abstract}

\begin{IEEEkeywords}
 distributed measuring, blockchains, legal metrology, software protection, security.
\end{IEEEkeywords}



% For peer review papers, you can put extra information on the cover
% page as needed:
% \ifCLASSOPTIONpeerreview
% \begin{center} \bfseries EDICS Category: 3-BBND \end{center}
% \fi
%
% For peerreview papers, this IEEEtran command inserts a page break and
% creates the second title. It will be ignored for other modes.
\IEEEpeerreviewmaketitle

\section{Introduction}
Measurement instruments (MI) are used in many application domains including industry, commerce, energy, transportation, health care and environment protection \cite{RodriguesFilho2015}. 
In Europe alone, MI are responsible for an annual turnover of more than 500 billion Euros \cite{Esche2015}. 
In developing countries, the demand for MI has increased substantially due to the adoption of technologies and methods well established in developed countries \cite{RodriguesFilho2015}. 
MI also can be seen as fundamental building blocks for new technologies such as internet of things and cyber physical systems (e.g., smart grids) \cite{RodriguesFilho2015,Esche2015,Camara2012,Boccardo2014,Peters2015,Oppermann2018}. 

MI are nowadays quite complex, since they are strongly based on embedded systems and are often connected and accessible by the Internet \cite{Esche2015,Camara2012}. 
%That creates
This kind of scenario might expose MI to security gaps that can be explored with malicious intent \cite{Boccardo2014,Peters2015}.
Legal metrology is responsible for promoting MI metrological assurance, establishing security requirements and technical activities such as type approval, verification and metrological supervision \cite{RodriguesFilho2015}.
However, the increasing complexity of MI affects such activities substantially. 
Type approval requires more effort while verification can involve use cases which are hard to reproduce inside labs. 
In turn, metrological supervision becomes difficult due to the diversity of MI models, their geographic spread and the limited resources owned by regulatory agencies.

We work with the hypothesis that the aforementioned difficulties should be overcome with alternative approaches that simplify MI design while employing strategies to decentralize metrological supervision. 
Such idea finds many aspects in common with a new trendy technology: \emph{blockchains} \cite{Nakamoto2008}. 
A blockchain can be described as a distributed data structure which assures information integrity and authenticity while providing a platform for executing self-enforced software procedures, called \emph{smart contracts} \cite{Christidis2016}. Blockchain solutions have been very successful in financial applications (e.g., Bitcoin and Ethereum), inspiring its use in different applications and knowledge areas \cite{Nakamoto2008,Christidis2016}.
More recently, a few works have proposed blockchain applications in legal metrology, which include decentralized audit, mechanisms for software loading, Public Key infrastructure (PKI) for MI manufacturers and Distributed Measuring Systems (DMS) \cite{Peters2018,MeloJr.2018a}.

In this paper we discuss how blockchains can improve measuring applications, evaluating two main aspects: \emph{distributed measuring} (DM) and \emph{decentralized surveillance}.
This paper extends the ideas presented in our previous work \cite{MeloJr.2018a}.
%under the title ``How Blockchains can improve Measuring Instruments Regulation and Control''
We start from preliminary concepts already consolidated in Legal Metrology about MI regulation and control. 
Then we explore ideas related to the integration of MI in DMS, proposing a blockchain-based model.
Such aspects result in an innovative concept that dissociates the measurement service from the measurement quantity while it improves MI security and makes metrological assurance simpler and less expensive. 

%Furthermore, this model impacts significantly attacks against MI since it restricts attackers capabilities.

Our main contributions can be summarized as follows:
\begin{itemize}
 \item We introduce the idea of DM using blockchains and describe its advantages when compared to traditional MI and other DM models. To the best of our knowledge, this is the first paper to describe a blockchains-based DMS.
 \item We propose an architectural model for implementing our idea, showing that MI and blockchains enable a new business model where the measuring process is an independent service, reducing conflicts of interest.
 \item We present a security analysis, showing that our model improves MI security since it constrains the attacker capabilities, thus simplifying MI regulation and control.
 \item We develop a practical case study using the Hyperledger Fabric \cite{Androulaki2018} platform. We create a blockchain network that implements legally relevant software using smart contracts for measuring vehicle speed. We also present results that demonstrate the feasibility of our idea.
 \item We point out challenges that shall be addressed in future works using blockchains in measuring applications.
\end{itemize}

%However, legal metrology have became more challenging due to the aggregation of new technologies and their increasingly complexity. Following an inevitable trend, measurement instruments are nowadays strongly based on embedded electronic and software. Oftentimes they are connected by networks and even can be accessed on internet, creating security gaps that can be explored for malicious use. Attacks against measurement instruments can look after different goals. The majority are associated to economic undue advantages. Measurement frauds in smart meters, usually classified as non-technical losses, are very common in developing countries \cite{DeFaria2014}. The same happens with scales, fuel pumps, and a diversity of instruments related to the commerce of measured goods \cite{RodriguesFilho2015}. Attacks can also intend to steal sensitive information and intellectual. In some cases they even threat people physical integrity, tampering measurements related to medical procedures or vehicles control, for instance.

%Aiming to avoid and preventing such threats, legal metrology has adopted a set of standards pointing good practices for creating, deploying and inspecting measurement instruments controlled by electronic and software \cite{Esche2015,Peters2015}. Besides to propose requirements for hardware, software and communication protection in MI, these standards also indicate guidelines for two-level activities in legal metrology \cite{RodriguesFilho2015}, which are:
% One can mention WELMEC Software Guide 7.2 in Europe Union and OIML D31 (E) from International Organization of Legal Metrology, taking as reference for security requirements in several countries \cite{Peters2015}. In addition, software quality standards already used in more conventional use cases, such as ISO/IEC 27001 and ISO/IEC 15408 (Common Criteria), have been applied in the control of measuring instruments \cite{Esche2015}.

%What has been seen in recent years is that activities related to type approval, verification and surveillance of MI have became more expensive. Due to hardware and software complexity, type approval can require more effort at same time that verification can involve use cases which are hard to reproduce out of the application field. In turn, surveillance became ineffective owing to the big number of measurement instruments and the capillarity of their distribution in different places. Initiatives involving surveillance are usually developed in the interest of govern agencies that no dispose enough resources to implement inspection in all deployed instruments. We believe such constraints should be overcome also by the use of new technologies. In this context, alternatives that simplifies instruments hardware and software, combining with strategies for decentralize efforts related to metrological supervision can be decisive. 

%In our studies, we found that blockchains can play an important role on improve legal metrology activities. That can be done assuming a conceptual model for implementing MI where three main aspects are reached: separation of measured quantity and measurement service, distributed measurement and decentralized surveillance. In the next sections, we present an overview about how these three features can be put in practice using blockchains as well as an idea about how this conceptual model can impact legal metrology. Finally, we present potential applications that have been focus of our recently research efforts.

\section{Background}
\subsection{Legal metrology and MI reliability}
\label{s:mi_realiability}
Legal metrology embraces MI regulation and control. It is crucial to assure the correctness of measurements \cite{RodriguesFilho2015} and regulate consumer relations \cite{Oppermann2018}.
%, having a significant impact over countries competitiveness \cite{RodriguesFilho2015}. Legal metrology protects the economic system and regulates consumer relations, while enhancing MI public reliability \cite{Oppermann2018}.
Usually, legal metrology regulations are defined by government agencies or international committees. 
Regulation directives traditionally establish a set of requirements and activities \cite{RodriguesFilho2015}.
The main ones are related to legal control of MI and type approval, which can include documentation and code inspection, validation and verification; and metrological supervision, including quality, market and field surveillance.
%Activities are classified in two levels. The first one is related to legal control MI type approval, which can include documentation and code inspection, validation and verification. The second level activities covers metrological supervision, including quality, market and field surveillance. 
These activities are usually executed by notified bodies\footnote{Notified bodies are public or private parties organized for verifying MI.} that are designated to assert MI conformity \cite{Esche2015,Oppermann2018}. 
%In some cases, manufacturers are authorized to declare their MI conformity \cite{RodriguesFilho2015} although such practice is not recommended mainly in developing countries where malicious manufacturers can further measurement frauds \cite{Camara2012,Boccardo2014}. In practice, regulation directives should be formated considering risk scenarios and attacks related to the use of different MI in different places.

Legal metrology activities related to electronic and software controlled MI can demand more complex procedures and specialized knowledge.
Usually, regulation adopts security requirements and good practices from well-known technical standards \cite{Esche2015,Peters2015,Luchsinger2008}. 
The \emph{OIML D 31(E)} document \cite{InternationalOrganizationofLegalMetrologyOIML2008} and \emph{WELMEC Software Guide 7.2} \cite{EuropeanCooperationinLegalMetrologyWELMEC2015} are probably the more widespread standards for software-controlled MI design, deployment and inspection.
%Both documents are taken as reference by metrological agencies, notified bodies and manufacturers in different countries \cite{Camara2012,Peters2015}.
%Attacks against MI can look after different goals.
%\footnote{OIML is the International Organization of Legal Metrology.} 
%\footnote{WELMEC is the European committee to promote cooperation in the field of legal metrology.}

The majority of security issues with MI arise from parties seeking undue economic advantages.
%are associated to economic undue advantages.
%taken by attackers, which are interested in the measured physical quantity value.
A classical example occurs in the commerce of measured goods where vendors and consumers have conflicting interests \cite{RodriguesFilho2015}.
Malicious vendors can try to maximize profits while malicious consumers can try to minimize prices by frauding measurements.
Measurement frauds against MI (such as scales, energy meters and fuel pumps) are very common in developing countries \cite{Camara2012,Luchsinger2008}.
Attacks can also intend to steal sensitive information and intellectual property \cite{Camara2012,Oppermann2018}.
In some cases they even threaten people's physical integrity (e.g., tampering measurements related to medical procedures) \cite{Boccardo2014}.

\subsection{Distributed measuring}
Distributed measuring (DM), where components are connected through a network, is well studied.
%concept supported by previous works and correlated ideas.
Boccardo et al. \cite{Boccardo2014} describes a strategy to simplify MI type approval and supervision activities related to medical MI.
Their proposal consists in signing sensing raw data of a sphygmomanometer immediately after analog-to-digital (AD) conversion.
Although it is not a DM case, this approach suggests that part of the measurement computing can be done externally to the sphygmomanometer hardware core due to the use of a digital signature to check sensing data integrity and authenticity.
Peters et al. \cite{Peters2015} describes a MI security framework using virtual machines to separate \emph{legally relevant} (LR) and \emph{non-legally relevant} (NLR) software.\footnote{OIML D 31(E) and WELMEC 7.2 use LR to designate any component which can affect measuring final results, while NLR cannot do that.} The authors propose different virtual machines to execute LR and NLR functions and define secure interfaces for communicating among them. This approach is presented as an alternative to improve security and reduce MI complexity. Additionally, it enables virtualization using different hardware cores and consequently allows the implementation of \emph{Distributed Measuring Systems} (DMS). Lastly, a DM architecture using cloud computing is discussed by Oppermann et al. \cite{Oppermann2018}. The authors present advantages related to IT infrastructure cost-savings and the possibility of MI manufacturers to offer modern interconnected devices and features.
They also present a comprehensive example of how to integrate energy meters in a DMS. 
In contrast, they also point out issues related to communication security, data management, and reliability. Roughly speaking, they assert that the following challenges need to be addressed:
\begin{itemize}
 \item DM instruments must be as secure as their classical counterparts.
 \item Large amounts of data will be accumulated in distributed repositories, requiring proper treatment.
 \item If distributed service providers are considered untrustworthy, then data security is very difficult to assure. 
\end{itemize}

\subsection{Blockchains}
Blockchain is an emerging technology which has caught the attention of stakeholders in different industry segments. 
Initially associated with crypto-currency markets due to Bitcoin popularity \cite{Nakamoto2008}, blockchain-based architectures have been proposed for a wide set of application areas, including sensor networks, internet of things, smart cities, among others \cite{Christidis2016}. 
%become one of the most promising technologies for the next generation of Internet interaction systems, such as smart contracts, public services, Internet of Things (IoT), reputation systems and security services

Conceptually, a blockchain can be regarded as a distributed append-only data structure (designated as \emph{ledger}) which is replicated and shared among a set of network peers \cite{Christidis2016}. 
This structure consists of a sequence of blocks where block $n$ is cryptographically linked to the block $n-1$ using a hash function.
Consequently, block $n$ cannot be changed without also modifying all subsequent blocks $n + i, ..., n + k$ \cite{Sousa2018}. 
Being a decentralized model, blockchains availability does not depend on third parties, which can greatly save costs.
%\cite{Zheng2017}.
In turn, integrity and availability are ensured by consensus among the peers, preventing the whole chain from being modified and requiring an agreement about any block to be appended to the ledger \cite{Sousa2018,Vukolic2016}.
Blockchain platforms can be classified as \emph{permissionless}, in which anybody can join and participate in the network consensus, or \emph{permissioned}, in which consensus is achieved by a set of known and identifiable peers \cite{Vukolic2016}. Usually, permissioned blockchains consensus protocols expend less computational resources and can reach better transaction latency and throughput.
%\cite{Sousa2018}.
%REMOVIDO: One can say that blockchain enables trustless networks while using public key cryptographic for providing confidentiality and authenticity. 

A blockchain can store virtually any digital asset, from data to self-executing scripts, usually defined as \emph{smart contracts}. %Ethereum \cite{Christidis2016} is probably the most well-known blockchain implementation supporting that. 
This makes blockchains not only a data storage solution but also a complete distributed platform for proper and distributed automated workflow \cite{Christidis2016}.
Once smart contracts are executed at every network peer in an independent and automatic manner, software integrity is achieved from blockchains integrity as a whole. 
%Such property has a remarkable impact over applications that require software identification and integrity checking.

\section{Defining MI Security Scope}
In this work, we want to evaluate the security level of different measuring systems models and point out the advantages and drawbacks of these models. 
We are especially interested in MI reliability and the required effort for providing MI metrological assurance.
Metrological requirements and activities are very particular for different MI classes. 
However, a simple set of requirements and activities are representative of most software-controlled MI concerning security properties. 
From that point of view, we define our MI security scope based on a generic attack model and its respective metrological assurance framework. 
We described both in this section.
%Both consist of a minimal set of requirements and activities necessary for providing MI realiability.

\subsection{Attack model}
\label{s:attackmodel}
We consider a simple attack model that can be built from MI use cases, according to OIML D 31(E) and WELMEC 7.2 guides. MI are targeted by malicious entities trying to get undue economic advantages by tampering with measurements. 
Basically, the attacker capability consists of changing the MI expected behavior, tampering any LR component and compromising the reliability of the measurements.

We assume the attacker could be any entity with access to the MI components or sensitive features, at any moment of its lifecycle. 
Attackers can be manufacturers, vendors, clients, and other entities. 
Malicious manufacturer staff (e.g., a malicious programmer) can inject software vulnerabilities and backdoors, "selling" them to other potential attackers. 
Once an MI is deployed, vendors and clients can have access to its resources, exploring eventual failures and misbehaviors or changing sensitive parameters related to MI accuracy. 
Furthermore, modern MI usually provide interfaces for loading software updates and upgrades. % and improvements.
Such features can be explored by malicious vendors and clients for loading tampered LR software or for modifying critical MI parameters.

Conversely, we establish that an attacker cannot compromise tamper-proof hardware devices, neither cryptographic primitives and communication protocols from algorithms recognized as secure. Also, we also define that an attacker cannot take part in collusion attacks with more than a fraction of peers that integrates the network. The exact value of this fraction depends on the blockchain implementation \cite{Vukolic2016}. %This last restriction is a necessary condition for claiming blockchains as secure architectures.  

\subsection{Basic Metrological Assurance Framework (BMAF)}
We assume the existence of a Basic Metrological Assurance Framework (BMAF) tailored to implement MI regulation and control, which works as a countermeasure to the previously described attack model. Such BMAF gives a minimal set of requirements and activities, which is very realistic since its statements can be found in regulation directives implemented in several countries \cite{Esche2015,Camara2012,Boccardo2014,Luchsinger2008}.

Our BMAF sets the following protection requirements:
\begin{itemize}
 \item \textbf{R1}: MI have reliable physical sealing to protect physical components such as sensors and electronic circuits;
 \item \textbf{R2}: MI implement acceptable mechanisms for LR software identification and integrity checking by notified bodies during MI supervision;
 \item \textbf{R3}: MI implement security mechanisms for LR software loading that accept only software modules signed by manufacturers and responsible notification bodies.
\end{itemize}

In turn, BMAF also establishes the following control and supervision activities:
\begin{itemize}
 \item \textbf{A1}: MI hardware and software detailed analysis, LR software source code inspection and conformity assessment regarding the MI protection requirements;
 \item \textbf{A2}: MI validation and verification of all relevant MI use cases identified during type approval;
 \item \textbf{A3}: MI supervision by periodic inspection in both manufacturing site and application field. Activities must include MI seal verification and LR software identification and integrity check.
\end{itemize}

\section{Blockchains in Measuring Systems}
In this section, we compare three different measuring system models: the traditional MI, a cloud-based measuring system and our blockchain-based measuring system model (Figure~\ref{f:compare}). For each model we describe the relevant supervision activities, using different sized icons to represent the expected magnitude of effort and cost associated with it.

\begin{figure*}[!t]
\centering
\includegraphics[width=.69\textwidth]{measuring} %[width=2.5in]
\caption{Comparing measuring models: (A) Traditional MI; (B) Cloud Measuring System; (C) Blockchain Measurement System.}
\label{f:compare}
\end{figure*}

\subsection{Traditional MI}
\label{s:mi_traditional}
Traditional MI can be seen as dedicated computers calculating measurements of a physical quantity (e.g., size, weight, speed). They include sensors for interfacing with the physical world and AD converters for gathering data, besides other LR and NLR components, which are usually software modules. Sensors and AD converters are also LR components, being usually immutable hardware components (Figure~\ref{f:compare}-A).

Although LR and NLR software separation is a well-known concept, many MI manufacturers do not adopt such a practice.
The claimed reasons are costs, computational resource restrictions or the existence of legacy software. 
However, despite their complexity, traditional MI software modules are usually monolithic systems \cite{Abreu2017}. 
This fact affects metrological assurance activities substantially, making MI regulation and control more expensive and complex, due to the following aspects:
\begin{itemize}
\item Type approval can demand MI hardware and software evaluation and check against a set of integrity requirements. Since LR and NLR are usually tightly coupled, notified bodies leading type approval need to evaluate and attest the compliance of all software modules. In some cases, LR software source code must be inspected for assuring their correctness.
\item Software validation and verification can become more difficult due to the diversity of MI use cases, many of them being hard to reproduce out of the real measurement environment. 
\item Metrological supervision requires notified bodies to have sufficient staff to proceed with MI surveillance activities in both manufacturing and the field. Although physical seals can be helpful to protect physical components, they are ineffective for protecting software components.
%For instance, fuel pumps supervision requires technicians able to inspect electronics components and identify the ones that do not belong to the original design. Software introspection \cite{Boccardo2014} and platform attestation \cite{Peters2015} are approaches used for providing software identification and integrity check.
\end{itemize}

Due to their complexity, the activities also demand a highly qualified professional profile, complementary checking, and greater supervision staff proficiency. These factors contribute to make MI regulation and control a very expensive and time-consuming process.

\subsection{Cloud-based Measuring System}
\label{s:mi_cloud}
When LR and NLR components are properly separated into independent modules, one could run these modules in different devices connected by well-defined interfaces.
Such architecture leads to a DMS.
For evaluating its properties, we take a cloud computing MI model inspired by Oppermann et al. \cite{Oppermann2018}.
In this model, LR and NLR software are running as cloud services, outside of MI physical set (Figure~\ref{f:compare}-B).
We assume MI communicates with the cloud services using a secure channel (e.g., TLS) and that side channel attacks are infeasible.

When traditional and cloud models are compared, one can observe that the distributed architecture simplifies MI devices.
MI practically do not include software components anymore once both LR and NLR software modules are running in the cloud.
In practice, MI is now set up as a blend of sensors and AD converters.
A communication interface allows the MI to send sensing raw data to the cloud measuring system. 
Basically, the MI could be designed only based on hardware components (e.g., smart sensors, cryptographic chips), although one should consider that some simple software could be necessary.
In any case, a significant amount of software is moved from the MI to the cloud, being provided as a service. 
Consequently, LR and NLR software can be scaled accordingly to the demand.

A DMS easily enables software separation.
This happens because LR and NLR software do not run over a monolithic platform anymore.
That encourages manufacturers to decouple LR and NLR software, which consequently makes LR software less complex.
Such aspect also impacts LR software type approval efforts, saving time and costs associated with documentation analysis, code inspection, and testing.
%Despite of the challenges, we believe distributed measuring enables a trade-off among complexity and features in MI projects. In our conception, MI could adopt a minimalistic architecture, being composed only by sensors and analog-to-digital converters. All other features, including measurement computation, can be implemented in a distributed network. At the same time, legal metrology needs decentralized supervision capabilities once the resources dispose by notified bodies are scarce when compared with the number of deployed MI. We believe such goals can be achieved by distributed measuring systems using blockchains.

\begin{table*}[t]
\centering
\caption{Traditional MI, cloud model, and blockchain model security analysis summary.}
\label{t:sec_analysis}
\begin{tabularx}{1\textwidth}{|c|l|X|X|}
\hline
                                  & \multicolumn{1}{c|}{\textbf{Trad. MI}} & \multicolumn{1}{c|}{\textbf{Cloud Model}}                                                                             & \multicolumn{1}{c|}{\textbf{Blockchain Model}}                                                               \\ \hline
\multicolumn{1}{|c|}{\textbf{R1}} & Required                                     & Required (tamper proofing).                                                                                           & Required (tamper proofing).                                                                                  \\ \hline
\multicolumn{1}{|c|}{\textbf{R2}} & Required                                     & Required only for LR software in the cloud.                                                                           & Unnecessary, LR smart contracts have integrity enforced due to blockchain properties.                        \\ \hline
\multicolumn{1}{|c|}{\textbf{R3}} & Required                                     & Required only for LR software in the cloud.                                                                           & Unnecessary, LR smart contracts are signed by notified bodies and checked by blockchain peers on deployment. \\ \hline
\textbf{A1}                       & Necessary                                    & Necessary, but the evaluation of LR software in the cloud is expected to be easier than MI embedded software.         & Necessary, but the evaluation of LR smart contracts is easier than the other models.                         \\ \hline
\textbf{A2}                       & Necessary                                    & Necessary, but LR software use cases are reduced and its V\&V can be performed without the need of field tests.               & Necessary, but LR software use cases are reduced and its V\&V can be performed without the need of field tests.      \\ \hline
\textbf{A3}                       & Necessary                                    & Partially necessary, since periodical inspections take place only in data centers where cloud servers are hosted. & Unnecessary, LR smart contracts have integrity enforced due to blockchain properties.                        \\ \hline
\end{tabularx}
\end{table*}

\subsection{Blockchain-based Measuring System}
\label{s:mi_blockchain}
Now we introduce the \emph{blockchain model} (Figure~\ref{f:compare}-C). 
We consider that MI generate and store reliable measurements of physical quantities while managing the interests of different involved parties (e.g., consumption relations). 
Thus, measurements can be seen as \emph{transactions} whose values must be protected against tampering and accidental changes \cite{Esche2015}.
Such aspects make DM a typical use case for blockchains applications. 

As a first and intuitive insight, we devise a distributed ledger storing reliable measurement transactions which can be checked by any involved party. 
Also, the blockchain would support the execution of LR (and even NLR) software using smart contracts, which process information from sensors and generate a consolidated measurement value. 
The integrity of measurements and LR software (as smart contracts) is preserved by the blockchain inherent properties \cite{Zheng2017}.
The ledger accounting enables the management of cumulative consumption transactions, such as energy and gas metering.
If a financial blockchain platform is used, it can integrate billing and payment functions. 
That is an interesting additional resource when compared to the cloud model.

We can glimpse a practical implementation of such DMS in the following example related to energy measuring. 
Smart meters can be designed in a straightforward way: a tamper-proof hardware with only (1) voltage and current sensors; and (2) a module able to sign sensors' raw data and send them as a blockchain transaction. 
In the blockchain network, the peers invoke a smart contract that implements all remaining LR computation (e.g., signal processing, noise reduction, values integration). 
The blockchain ledger stores the final measurement. 
In turn, one obtains the cumulative energy consumption by querying each meter's stored measurements.

%In light of these models properties, we claim that distributed measuring systems can be built in both blockchain and cloud architectures. However 
There is a crucial difference between blockchain and cloud models: the liability of the distributed services. 
In most use cases, MI belong to one of the parties interested in the measurement computing result. 
Energy and fuel are typical examples where vendors of goods own the MI.
In the cloud model, one can expect that an interested party will hold the cloud measuring services. 
On the other hand, the blockchain is a truly decentralized architecture, being held potentially by several parties. 
Thus, one can expect that a blockchain model will require the contribution of different parties interested in the measurement activities, and consequently it will need to be designed following a different philosophy.

In the blockchain model, we devise measuring as \emph{a service offered by someone without any interest in the measured quantity}. 
This idea is remarkably distinct to the traditional scenario where a vendor provides MI for measuring and is rewarded proportionally to the measurement. 
This idea fits very well in the blockchain model. 
Smart contracts can be used for computing measurements based on sensing information. 
However, they are coded by different parties that do not have conflicts of interest related to the measured quantity.
Whatever the measurement result is, these parties shall be rewarded by a pre-set value. 
That motivates new players to provide better measuring algorithms.
%once as faster they execute, more credits they can earn.
Additionally, this strategy creates incentives for keeping the blockchain network since that becomes profitable. 
%One could argue that such concept is an innovative idea once it breaks with the manner MI are traditionally used in consumer relations.
This concept also breaks the traditional way MI are used in consumer relations, creating a new market for players who want to offer computing services for measuring. 

The blockchain model also enables a set of complementary activities involving MI market and field surveillance that can be done by checking measurements inserted in the distributed ledger.
Besides notified bodies, any entity representing society interests, consumers, goods providers, among others, can take part in additional supervision activities.
We call that \emph{public surveillance}.
Such efforts can include smart contracts for generating redundant measurements for counter-proofing, or statistical analyses against the ledger looking for fraud evidence or patterns, for instance.

A last important aspect is the intrinsic blockchain robustness against attacks and failures.
Since blockchains make extensive use of cryptography in both transactions and storing, information reaches a high level of protection regarding authenticity and integrity assurance.
The known security attacks that can compromise a blockchain network are related to collusion among the stakeholders that participate in the consensus decision \cite{karame2016bitcoin}.
However, as more organizations take part in the consensus, more expensive and unfeasible these attacks become.
Thus blockchains can provide a secure mechanism for assuring the legal liability and trustworthiness of instruments and measurements.

%at Section \ref{s:sec_analysis}.

%However, that is not the only advantage. We believe that three main aspects must be emphasize when all the scope of measuring and MI legal control are considered:

% MI can be seen as dedicated computers which calculate measures of a physical quantity (e.g., size, weight, speed). The MI reliability depends on the integrity of measure values, that means the measurement. A measurement is always an approximation of the real physical quantity, being always subject to an error, called uncertain. For each different classes of MI, legal metrology determines an uncertain threshold, which defines the MI precision.

% A measurement can be seen as a transaction. Such analogy is very intuitive in commercial transactions where goods prices are calculated using a MI (e.g., energy meters, fuel pumps, and scales). Attacks against MI usually try to compromise their precision, aiming to economic advantages (e.g, in commercial transactions where price is defined by weight, the seller wishes the scale points a higher measurement, while the buyer wishes the scale registers a lower measurement).

% Based on the aforementioned aspects, we can propose a conceptual model where MI are deployed in a blockchain architecture. MI are devices that generate transactions. Essentially, these transactions records a measurement involving one or more parties. Transactions will be sent to a distributed ledger and, once there, they can not be changed anymore. More than that, we assume that the follow aspects will be fulfilled: distributed measurement, separation of measured quantity and measurement service, and decentralized surveillance. We explain each one of them in details hereafter.

%Considering that, we propose a conceptual model where MI are deployed in a blockchain architecture. MI are devices that generate transactions. Essentially, these transactions records a measurement involving one or more parties. Transactions will be sent to a distributed ledger and, once there, they can not be changed anymore. Furthermore, we assume the following aspects will be fulfilled: %distributed measurement and decentralized surveillance. We explain each one of them hereafter.

% \subsection{Measured quantity \textit{versus} measurement service}
% In trading applications, MI usually belong to one of the involved parties. Energy and fuel are classical examples where MI are owned by the good vendors. Sometimes consumers also have physical access to the MI (e.g., smart meters are deployed at consumers house). Even when MI are properly protected, with sensitive features accessible only by manufacturers or supervision agents, MI ownership represents a security risk once it can motivate attacks and make them easier. A reasonable level of security can be obtained when distributed measurement is implemented. However, traditional approaches using cloud services keeps the measurement procedures under control of the vendor. 
% 
% A better solution is obtained when measuring is a service offered by someone without any interest in the measured quantity. This is remarkably distinct to the traditional scenario where a vendor provides MI for measuring and is rewarded proportionally to the measurement. One could say that such concept is an innovative idea once it breaks with the manner how MI are used in consumer relations and creates a new market for players who want to offer computing services for measurement.
% 
% The described model fits perfectly in blockchain architecture. Such implementation can result from the use of smart contracts for computing measurement based on information from sensors, just like described in the previous section. However the difference is that smart contracts are coded by different parties that do not have conflicts of interests related to the measured quantity. Whatever the measurement result, these parties are going to be rewarded by a fixed value. That motivates new players to compete by more efficient measurement algorithms once faster they execute, more credits they earn. At same time, such model creates an incentive for keeping the blockchain network since that becomes profitable. Furthermore, type approval is favored again, because it is restrict to the smart contract code inspection.

% The blockchain conceptual model has also an important advantage when compared to other distributed approaches: measuring services are offered by someone without any interest in the measured quantity. This is remarkably distinct to the traditional scenario where a vendor provides MI and is rewarded proportionally to the measurement. In trading applications, MI usually belong to one of the involved parties (e.g., fuel pumps are owned by fuel vendors). Sometimes consumers also can have physical access to the MI (e.g., smart meters are deployed inside consumers house). MI ownership and access represent a security risk once that can motivate and facilitate attacks. This problem persists in distributed scenarios using cloud computing as measurement procedures usually are running in machines under control of the vendor. However, when using blockchains, smart contracts are coded by different parties that do not have conflicts of interests related to the measured quantity. Whatever the measurement result, these parties are going to be rewarded by fixed values. That motivates new players to compete by more efficient measurement algorithms while, at same time, creating incentives for participating in the blockchain network. Moreover, type approval is favored again, because it is restrict to the smart contract code inspection.

\section{Security analysis}
\label{s:sec_analysis}
In this section, we present a security analysis by comparing the measuring models discussed in the previous section.
We consider the attacks and the metrological assurance framework BMAF described previously.
We demonstrate how the traditional MI and DMS impact BMAF requirements and activities.  
Table \ref{t:sec_analysis} depicts such analysis.

Initially, we evaluate traditional MI security. 
In this scenario, one should note that BMAF requirements and activities are necessary to prevent attacks.
As already discussed at Section \ref{s:mi_traditional}, the activities of control and supervision of traditional MI involve a substantial effort.
Documentation analysis, code inspection, and software validation and verification need to be done on all components and software modules.
In turn, supervision also requires experienced surveillance technicians to implement inspection and software integrity checks.

When the cloud model is analyzed, one can notice that MI become simpler because LR and NLR software are now running in the cloud.
Additionally, such situation reduces the capabilities of a typical attacker (e.g., consumers do not have physical access to MI software interfaces anymore).
The BMAF protection requirements are still necessary, however requirements R2 and R3 are applied on the LR software implemented in the cloud. 
Supervision activities are also impacted, requiring fewer efforts to be executed.
In A1, the document analysis and the code inspection of LR software running in the cloud are expected to require less effort than embedded software evaluation. 
Activity A2 is also made simpler once LR software tests can now be performed using interface stubs, without the need of real MI physical environment. 
Similarly, A3 also becomes less expensive because LR software identification and integrity check are executed against cloud servers, which are far fewer than the deployed MI. 
Finally, field surveillance for checking MI physical seals can also be eliminated. 
If we assume simplified MI as immutable instruments, they can be conceived as tamper-proof devices. 
That approach could eliminate the need for verifying MI seals as it implies that the MI will be permanently damaged and any attack trying to explore such vulnerability will not succeed.

Lastly, we evaluate the blockchain model. 
In addition to presenting the same characteristics of the cloud model, the blockchain security properties also affect BMAF requirements and activities. 
LR software is now a smart contract whose the deployment rules can be enforced for requiring developers and notified bodies attestation, something that automatically satisfies R3 and makes its regulation unnecessary.
Once deployed, LR software is distributed among the peers, and it cannot be changed anymore.
Blockchain peers cannot execute a different smart contract code. Otherwise, blockchain security assumptions will be violated.
In consequence, R2 also becomes unnecessary.
Regarding the activities, although A1 and A2 are still necessary, they should become much simpler when compared to the other models.
This happens because the structure of smart contracts significantly limits the complexity resulting from having different technologies, software components and programming languages while imposing software separation.
Finally, A3 becomes unnecessary in a blockchain network for the same reasons as R2.

We conclude that while DM already reduces attackers capabilities, such reduction is more accentuated in the blockchain model. 
Once LR software is produced by players who are exempted from conflicts of interest, many activities related to the assurance of software correctness and integrity are made simpler or even unnecessary. 
The blockchain security properties play an important role in this context. 

\section{Case Study}
\subsection{Speed meters and the case study scenario}
In this section, we develop an experiment to demonstrate the feasibility of our proposal.
It consists of a vehicle speed DMS using blockchains (Figure \ref{f:speedmeter}).
These meters are efficient solutions for estimating vehicle speed on public roads, generating traffic statistics and enforcing speed limits for drivers \cite{Ki2006,S2011}.
The meter detects each vehicle, determines its speed and captures one or more pictures identifying the vehicle's license plate when necessary.
The measurement and the license plate image constitute the legally relevant record, which is expected to be reliable and protected against frauds.

We choose the city of Sao Paulo, in Brazil, as a case study.
Sao Paulo has a vehicular fleet with more than 8 million vehicles and about one thousand speed meters deployed along its roads \cite{CompanhiadeEngenhariadeTrafego-CET2017}.
Furthermore, over the last two years, we have proceeded with formal type approval of speed meters in the Brazilian National Institute of Metrology, Quality and Technology (Inmetro\footnote{\url{http://www.inmetro.gov.br}}).
This experience provides valuable information that supports the analysis in this section.  

Developing countries are essentially guided for legal metrology policies related to fraud detection and avoidance \cite{RodriguesFilho2015,Camara2012,Luchsinger2008}.
In the majority of cases, these countries adopt restrictive legal metrology activities, which include detailed type approval processes and intensive metrological inspection, especially field surveillance.
That is the Brazilian reality regarding vehicle speed meters.
These instruments are under restrictive regulation and control directives, which associates them to a WELMEC 7.2 class-D risk level \cite{EuropeanCooperationinLegalMetrologyWELMEC2015}.
In this aspect, the blockchain-based DMS can introduce promising advantages, as it simplifies the metrological assurance framework activities, while preserving many of the advantages from a DMS in terms of performance and costs saving.

\subsection{Conceiving a vehicle speed meter DMS}
In Brazil, vehicle speed meters are usually built as \emph{Type-U instruments}.
WELMEC 7.2 \cite{EuropeanCooperationinLegalMetrologyWELMEC2015} defines this classification as instruments that run their software in universal computer hardware.
This happens because speed meters nowadays have an extensive list of requirements and aggregate a large number of NLR features.
Consequently, their software usually is quite complex and hard to evaluate and test.
Besides, speed meters manufacturers complain about opening their solutions for inspection due to intellectual property issues.
That is the case of the speed meters evaluated in Inmetro.
So they are representative cases of the traditional MI model described in this work.

Another aspect is that speed meter owners usually deploy their equipment at far places along roads spread over large geographic areas.
That increases the difficulty of regular inspection activities and consequently increases costs associated with metrological surveillance.
Thus, all those aspects make speed meters strong candidates for solutions that help to separate LR and NLR software.
%, while offering efficiencies for legal metrology activities such as documentation analysis, code inspection, V\&V, and field surveillance.

We implement such a solution creating a blockchain network for distributed measuring.
We integrate the speed meter's LR features in a simple tamper-proof hardware that uses two inductive sensors to capture the vehicle's magnetic profile \cite{S2011}.
After detecting a vehicle, this hardware uses a private key to sign the sensor's raw data and sends that to a blockchain-based DMS.
The blockchain executes the LR software as a smart contract and computes the vehicle speed.
The vehicle detection event also triggers any other device used for providing complementary evidence (e.g., a camera that captures the vehicle license plate).
Furthermore, manufacturers may aggregate any other module necessary for implementing NLR functionalities or even use the blockchain to do that.
Their decision does not affect the legal metrology activities once NLR features are not under regulation.

\begin{figure}[!t]
\centering
\includegraphics[width=.45\textwidth]{speedmeter} %[width=2.5in]
\caption{The blockchain-based vehicle speed DMS.}
\label{f:speedmeter}
\end{figure}

\subsection{Architecture using Hyperledger Fabric}
Our prototype uses Hyperledger Fabric \cite{Androulaki2018} as a permissioned blockchain, where the peers cooperate to store measurements and execute LR software.
Fabric is an open source blockchain platform that includes two concepts that are very helpful for implementing our idea: \emph{endorsers} and \emph{security policies}.

Endorsers are peers that effectively execute smart contracts, which are called \emph{chaincodes} in Fabric.
The way endorsers operate has significant implications concerning intellectual property and performance.
First, a manufacturer needs to reveal her LR software only to peers contracted to execute her measuring chaincode as a service, and to the notified body responsible for approving it.
Second, a manufacturer can size his solution performance by providing as many endorser peers as necessary, resulting in a scalable architecture.

Security policies can set the way the network validates the measurement provided by endorsers.
These policies can also work as a protection mechanism against collusion attacks and implement metrological surveillance.
Blockchain networks are constituted by peers that belong to different stakeholders or organizations.
Although they do not need to trust each other, each peer continually verifies other peers behavior.
Security policies define rules for such monitoring.
They can specify, for instance, which organizations must take part in a transaction endorsement (i.e., a measurement calculation).
The more organizations are involved, the more expensive it becomes to carry out a collusion attack.
Such monitoring activities are a sort of metrological surveillance since different organizations are continually checking the measurements resulting from a chaincode execution.

Since we resort to a permissioned blockchain, the consensus protocol plays an important role in our experiment.
Fabric refers to consensus as an \emph{orderer service}.
We perform our experiment with two different types of orderer services: the \emph{solo orderer} and the \emph{Byzantine Fault-Tolerant (BFT) orderer}.
The solo orderer service is native from Fabric distribution, and it is very practical to implement tests and develop proof of concept prototypes.
However, regarding security, the solo orderer cannot be considered a suitable solution because it implies that consensus comes from only one organization.
In turn, the BFT orderer \cite{Sousa2018} is fully replicated for tolerating Byzantine failures.
Thus, one can configure the BFT orderer with several replicas, with different organizations controlling each one of them.
Such approach employs a decentralized consensus provided by the BFT-SMaRt replication library \cite{Bessani2014}, thus providing security against collusion attacks.

\subsection{Describing MI regulation and control activities}
We set up our vehicle speed DMS as illustrated in Figure~\ref{f:scheme}.
Inmetro and LaSIGE represent two independent organizations with distinct functions.
Inmetro is a notified body responsible for regulating and controlling such instruments.
LaSIGE provides computational resources for executing LR software from speed meters.

The speed meter manufacturer implements LR software as a chaincode. 
%Fabric supports chaincode written in the Go programming language.
%Figure X shows a simple LR algorithm for calculating speed example in Go.
We create a chaincode written in Go that analyses raw data from both sensors and finds the moment when the vehicle activates each sensor.
Once the samples are taken in regular periods, and we know the distance between the sensors, it becomes trivial to determine the vehicle speed.
When the meter also enforces speed limits, one or more images from the vehicle's license plate can be necessary.
Such requirement brings some concerns about privacy.
Although the image is part of the legally relevant information, it is not necessary for determining the vehicle speed.
So one can use different approaches to avoid problems with privacy.
One idea consists of encrypting the images using asymmetric cryptography before sending them to the blockchain.
One can do that using the public key of the legal authority responsible for issuing traffic tickets.
A better alternative is to send only the image digital signature to the blockchain.
The image is kept in a private data storage, and the blockchain can attest its integrity and authenticity whenever necessary.
Such approach improves performance and eliminates privacy concerns.
We consider this approach in our solution.

Our experiment assumes the implementation of legal metrology activities as follows.
Firstly, the notified body proceeds with the MI type approval.
He does that by evaluating the MI device (i.e., inspecting sensors, cryptographic and communication features) and the LR source code (i.e., the Fabric chaincode).
After, the notified body executes the applicable tests for assuring all MI LR functionalities.
Once MI hardware and software are approved, the notified body is responsible for instantiating the chaincode in the blockchain.

In Fabric, a chaincode instantiation includes the notification of endorsers that will execute that chaincode.
To do that, the notified body inserts into the blockchain a new transaction comprised by the chaincode fingerprint (i.e., the software image hash) and its respective security policy.
Thus all the peers in the network know how to validate the chaincode execution by checking the applicable security policies.
After instantiation, the chaincode fingerprint becomes immutable, following the intrinsic blockchain properties.
Any future update in the chaincode will require a new instance of it, which means to create a new chaincode version.

After Inmetro instantiates the LR chaincode, any MI owner can contract computing services from the LaSIGE organization for executing the respective LR software.
In practice, LaSIGE provides endorser peers.
One must note that LaSIGE is only one possible organization that can offer such a service.
Although we have only two organizations in our experiment, a real scenario can include several independent organizations.
Each one of them could have endorser peers offering measuring services.
Once the MI owner finds an available endorser peer, she needs to install the approved LR chaincode.
That enables any MI in the field to generate transactions, ask endorsers to determine the vehicle speed, and store the vehicle speed legally relevant information in the blockchain ledger.

\begin{figure}[!t]
\centering
\includegraphics[width=.45\textwidth]{hlfscheme} %[width=2.5in]
\caption{Vehicle speed DMS solution scheme.}
\label{f:scheme}
\end{figure}

\subsection{Security analysis}
We verify how our implementation offers countermeasures against common attacks associated with the capabilities described at the Section~\ref{s:attackmodel}.
These countermeasures match the properties already discussed in Section~\ref{s:sec_analysis}.
%We describe some attacks and how the implementation using a simple MI conception and Fabric blockchain addresses them.
In the following we describe some attacks and how our blockchain-based DMS deals with them.

\subsubsection{An attacker tries to compromise the speed meter hardware integrity}
We conceive the speed meter hardware as a tamper-proof device.
Consequently, any attempt at violating seals or stealing a private key shall destroy the speed meter hardware, forcing its replacement and exposing the attacker.

\subsubsection{An attacker tries to install a malicious chaincode in an endorser peer}
The blockchain ledger contains the fingerprint of every instantiated chaincode, and legitimate endorsers automatically check the LR chaincode integrity on deploy time.
If a malicious speed meter owner tries to install a modified LR chaincode version, the endorser refuses such software.
Furthermore, the endorser appends a transaction registry in the blockchain.
That can be useful in audits for detecting attacks against software integrity in the future.

\subsubsection{An attacker can collude with an organization for loading malicious chaincode in endorser peers}
Security policies assure protection against collusion attacks.
For instance, suppose that an attacker colludes with LaSIGE, making compromised peers accept a modified LR chaincode.
One can avoid such attack by enforcing security policies defining that at least N peers of different organizations must endorse the chaincode execution.
Thus, a collusion attack including only the LaSIGE organization will not succeed.
The attacker needs to compromise more organizations, which makes the attack too expensive and, consequently, unfeasible.

\subsubsection{An attacker has success in colluding with enough peers for injecting fraudulent measurements in the blockchains}
One should remember that, in the proposed DMS, organizations compute measurements as an independent service (i.e., they do not have any advantage or reward regarding the measurement result).
Such business model discourages collusion and makes these attacks disadvantageous.
Even so, assuming that an attacker succeeds in compromising a sufficiently large number of peers for endorsing a malicious transaction, one can expose such fraud by auditing the measurement record.
Since Inmetro keeps any approved LR chaincode and the blockchain records all information used in any transaction, Inmetro can re-execute the LR chaincode and compare the obtained measurements.
The sensor raw data can easily have its integrity verified by using the MI public key.
Regarding the MI private key integrity, we already discussed this issue in the first attack described in this section.


\section{Performance issues}
An essential step in our experiment is the speed meter DMS performance evaluation.
We try to estimate the blockchain peers behavior without considering network communication issues.
Essentially, we are interested in two main aspects:
\begin{itemize}
   \item{The throughput and latency within each endorser peer. This subject is important because it helps to estimate how many peers a speed meter owner will need, considering a specific demand.}
   \item{The throughput of a simple blockchain network configuration. We test a high number of transactions against a network consisting of only two peers (one of them as an endorser) and two different types of consensus service.}
\end{itemize}

\subsection{Demand goals}
We estimate the experiment demand based on vehicles traffic real data.
According to a technical report from the Sao Paulo's Traffic Engineering Company 
(CET-SP) \cite{CompanhiadeEngenhariadeTrafego-CET2017}, there was in 2016 approximately one thousand speed meters spread along the city.
The same report analyses the vehicles flow on the main roads in the city and points out an average of 2,772 vehicles/hour (or 0.76 vehicles/sec) during rush hour.
Assuming such demand for a total of 1 thousand speed meters, we need a blockchain network able to process something around 800 tps (transactions per second).
Androulaki et al. \cite{Androulaki2018} benchmarks Fabric performance in something between 2,000 to 3,000 tps.
However, they consider specific scenarios with proper customizations for evaluating particular performance issues.
Our experiment does not employ any specific customization.
We use Fabric just as provided by its developers, in an ordinary hardware infrastructure available in any datacenter.
%That is done on purpose
The objective is to evaluate the results that a speed meter solution owner can obtain by using Fabric.

\subsection{Test environment setup}
Our blockchain network environment consists of 3 nodes from a Dell PowerEdge R410 cluster.
Each node has two CPUs Intel® Xeon® Processor E5520 with 2.27 GHz and 32 GB of RAM.
Fabric standard distribution version 1.1\footnote{\url{http://hyperledger-fabric.readthedocs.io/en/release-1.1}} runs over \emph{docker containers}, so that each physical node can host several peers.
For the sake of simplicity, we use three nodes for allocating the orderer service, the LaSIGE peers, and the Inmetro peers, respectively.

Regarding the orderer service, we test two different scenarios.
The first one is the native Fabric's solo orderer service, which is provided primarily for testing.
%and does not have a practical use.
The second scenario uses the orderer service developed by Sousa et al. \cite{Sousa2018}, which tolerates Byzantine failures.
%We configure this service to work with four replicas.

The client transactions load is mimicked using four nodes from a Dell PowerEdge R300 cluster, each one with an Intel Xeon Processor L5410 with 2.33 GHz and 8 GB of RAM.
The transactions simulation uses real data from speed meters developed by the company Perkons\footnote{\url{http://www.perkons.com}} SA, which kindly granted a dataset for this experiment.

\begin{figure}[!t]
\centering
\includegraphics[width=.45\textwidth]{chartsolo} %[width=2.5in]
\caption{Throughput and latency using native Fabric solo orderer.}
\label{f:latth}
\end{figure}

\subsection{Test methodology}
The performance tests execute as follows.
Each physical machine creates the respective Fabric docker containers (peers or clients).
Client containers are responsible for generating transactions.
We use Fabric as an \emph{off-the-shelf solution}, without any customization.
Each client instance corresponds to a container process that sends transactions to the blockchain.
%continually, which requires massive hardware resources.
We try to produce a maximum workload by increasing the number of clients.
%Reaching such conditions, we determine the blockchain's best throughput and latency.

%At the test execution, each client tries to produce a workload of $n$ concurrent transactions per second.
%In fact, there are some difficulties in effectively generating this number of concurrent transactions without modifying the Fabric implementation.
% Even so, the measured throughput and latency are precise once they comprise the receiving of all simulated transactions by the orderer service.

When a client instance is created, it selects a vector of bytes containing the sensors raw data from the dataset mentioned above, for each transaction.
Clients use the Fabric protocol to invoke an LR chaincode and send such data as an argument to an endorser peer from LaSIGE organization.
The endorser peer receives the vector of bytes, calculates the vehicle speed, and returns an endorsed package with the respective measurement.
The client uses such package for composing the complete transaction and sends it to the orderer service responsible for generating the ledger blocks and disseminate them to the other peers.
The client also keeps records of timestamps and the elapsed time for completing the transaction.
Such information gives the throughput and latency of the system. %estimates for the endorsement and orderer services.
%In turn, an independent peer in Inmetro organization provides the blockchain with throughput estimative. 
%We keep checking this peer once every 10 seconds, counting how many transactions were committed to the ledger.

% We also test scenarios with different orderer services.
% The first scenario uses the solo orderer service provide by Fabric.
% The second scenario works with the BFT orderer service with four replicas.
% We consider that each replica is under control of a different organization.
% The comparison between these scenarios is important for evaluating performance aspects.

% \begin{figure}[!t]
% \centering
% \includegraphics[width=.45\textwidth]{chartbft1r} %[width=2.5in]
% \caption{Throughput and latency using BFT orderer with one replica}
% \label{f:latthbft1}
% \end{figure}

\subsection{Performance test results}
Figures \ref{f:latth} and \ref{f:latthbft4} depict our tests results for the system using the solo orderer and the BFT orderer with 4 replicas, respectively.
With the solo orderer service, throughput and latency reach a better trade-off around 300 tps and 1 second, respectively.
The workload necessary to get such results corresponds to 600 simultaneous clients.
With a higher workload, performance degrades substantially.
We reach a throughput of around 260 tps and a high latency of 12 seconds when testing a workload from 1,200 clients.
This high latency can be explained by the transactions queuing in the solo orderer.
%A strategy of queuing transactions in the solo orderer can explain the high latency.
%However, this method compromises performance, resulting in low throughput and misusing of the MI communication channel.

The BFT orderer service with four replicas performs better.
It reaches the best trade-off with a throughput of 380 tps and a latency in 1.6 seconds with the same workload of 600 simultaneous clients.
The BFT orderer also keeps throughput stable at about 360 tps even with a workload of 1,200 clients, presenting only a slight increase of 1 second in latency.
The BFT orderer optimizes latency by discarding the exceeding of transactions after reaching its max throughput.
However, clients need to control refused transactions, creating their queue and resending the transaction again after some time.

\begin{figure}[!t]
\centering
\includegraphics[width=.45\textwidth]{chartbft4r} %[width=2.5in]
\caption{Throughput and latency using BFT orderer with four replicas.}
\label{f:latthbft4}
\end{figure}

The BFT orderer service also includes an important aspect.
It enables a truly distributed consensus service, once each replica belongs to a different organization.
Although the number of replicas impacts performance \cite{Bessani2014}, it aggregates security by preventing collusion attacks.

One can observe that our results point out a difficulty in dealing with the estimated peak demand of 800 tps.
However, we understand that the blockchain can absorb such demand along the day since the number of transactions goes down after the rush hour.
We conclude that Fabric performs satisfactorily for implementing a speed meter DMS.
Furthermore, if necessary, one can even reach a better performance by customizing Fabric features, something indeed feasible once the platform is an open source software product.

\section{Quantitative Assessment}
In this section, we provide some quantitative assessment regarding the adoption of the blockchain-based DMS model, when compared with traditional MI.
We do that by evaluating advantages and drawbacks related to both technologies (Table~\ref{t:quantitative}).
The discussions present in the section are not exhaustive.
Actually, they are preliminary results of a risk analysis study in progress at the moment.
However, we believe that the discussed aspects are useful for providing assessment information to people interested in implementing a blockchain-based DMS.

\subsection{Software vulnerabilities}
There is a direct relation between the size of a software product and the number of software defects.
Alhazmi et al. \cite{Alhazmi2007} estimate that the ratio of remaining vulnerabilities to the total number of software defects is often in the range of 1–5\%.
Considering that, we estimate how much the amount of LR software can impact the security of traditional MI and blockchain-based DMS.
In our analysis, we adopt the average ratio of approximately 6 D/kLOC (defects by thousands of lines of code) reported by Carrozza et al. \cite{Carrozza2015}.
In our experience with speed meters type approval at Inmetro, we found that LR software average size is around of 3,000 LOCs.
This software size statistically suggests the existence of about 18 software defects and, consequently, a high probability of having a vulnerability.
Such LR software size is a consequence of the strong coupling between LR and NLR modules.
When one considers only the software effectively used in measuring, the LR software size can be remarkably reduced.
We confirm that in our speed meter DMS implementation.
Its LR software consists of a Go language chaincode that requires no more than 200 LOCs.
Such size points out an estimate of no more than two remaining software defects, which also reduces the chances of potential vulnerabilities.

% \begin{table}[t!]
% \centering
% \caption{Advantages and drawbacks of the Traditional MI and our Fabric DMS implementation.}
% \label{t:quantitative}
% \begin{tabularx}{0.47\textwidth}{|@{}l|c|c@{}|}
% \toprule
% \textbf{Evaluated aspect}             & \textbf{Traditional MI} & \textbf{Fabric DMS} \\ \midrule
% Expected software defects (D/kLOC) & 18                      & 2                   \\ \midrule
% Code inspection effort (men/month)    & 0.5                     & 0.033               \\ \midrule
% Costs with hardware\protect\footnotemark (U\$)             & 500,000                 & 5,000               \\ \midrule
% Connectivity dependency               & None                    & Very high           \\ \midrule
% Estimative of availability            & Fair                    & High              \\ \bottomrule
% \end{tabularx}
% \end{table}
% %\addtocounter{footnote}{-1}
% \footnotetext{The analysis do not include the hardware required for NLR software.}

\begin{table}[t!]
\centering
\caption{Advantages and drawbacks of the Traditional MI and our Fabric DMS implementation.}
\label{t:quantitative}
\begin{tabularx}{0.48\textwidth}{|l|c|c|}
\hline
\textbf{Evaluated aspect}             & \textbf{Trad. MI} & \textbf{Fabric DMS} \\ \hline
Expected LR software defects (D/kLOC) & 18                      & 2                   \\ \hline
Code inspection effort (men/month)    & 0.5                     & 0.033               \\ \hline
Costs with hardware\protect\footnotemark (U\$)             & 500,000                 & 5,000               \\ \hline
Connectivity dependency               & None                    & Very high           \\ \hline
Estimative of availability            & Fair                    & Very high           \\ \hline
\end{tabularx}
\end{table}
\footnotetext{The analysis do not include the hardware required for NLR software.}


\subsection{Code inspection efforts}
The LR software size also affects the efforts required by MI type approval, especially code inspection activities.
We evaluate such impact by using the studies of Ebert and Jones \cite{Ebert2009} about the quality of embedded software.
Their work reports an average production rate of 60 FP (Function Points) by men/month in code inspection.
Since the majority of the manufacturers implement their software in C language, we adopt the FP/kLOC conversion rate in the QSM Function Point Languages Table \footnote{http://www.qsm.com/resources/function-point-languages-table} of approximately 100 LOCs per FP.
Considering the LR software sizes estimated previously, we have an expected code inspection effort of  0.5 men/month in the traditional MI model against 0.033 men/month in the blockchain-based DMS.

\subsection{Costs with hardware}
Other important aspect concerns the costs associated with the Type-U hardware adopted by speed meters manufacturers.
In Brazil, due to the high temperatures associated with the tropical climate, manufacturers need to build their meters using specific motherboards and components.
Such hardware easily exceeds U\$ 500, something that makes the Brazilian traditional speed meters a quite expensive product.
In our experiment, we consider the deploy of 1,000 meters, which correspond to the number of such devices in Sao Paulo.
With the traditional MI model, that implies a direct cost of approximately U\$ 500,000 only with the Type-U hardware.
In our implementation using Fabric, we succeeded in executing the LR software of the same number of MI in only three nodes of a Dell PowerEdge R410 cluster, which represents a cost with hardware that does not exceed U\$ 5,000.
We cannot directly compare both scenarios because manufacturers avail the Type-U hardware deployed in the field to provide NLR features.
However, we understand that the NLR software can also be provided by independent remote services.
Furthermore, the required hardware becomes less expensive once remote services run from data centers where the environment do not present the same inclement weather found in the field.
Thus cost saving is evident in the adoption of a distributed and decentralized solution.

\subsection{Connectivity dependency and demand}
Connectivity is a critical requirement in the blockchain-based DMS.
The simple MI hardware needs to send sensing information to the blockchain on every vehicle detection.
If the device loses connectivity, the information must be discarded or stored in temporary memory.
Such scenarios can require a more sophisticated MI hardware (e.g., additional memory for temporary data and a state machine to deal with connectivity restrictions) or even compromise MI availability.
On the other hand, the traditional MI includes enough computational resources to manage information when there is no connectivity.
They can even operate offline for several days without any problem related to information loss.

Concerning the demand by connectivity, we can state that both solutions are similar.
Although the traditional MI can send information in batch mode, the expected amount of information is practically the same as in the blockchain-based DMS.
Furthermore, despite the extensive use of cryptography in a blockchain application, that does not represent a significant overhead regarding the amount of propagated information.
Roughly speaking, the use of connectivity resources depends on the number of detected vehicles and not from the adopted model.

\subsection{Service reliability and availability}
We evaluate the reliability and availability of the services provided by both speed meter models by comparing properties of a centralized and a distributed system.
In Reliability Theory, the MTBF (Mean Time Between Failures) and the MTTR (Mean Time to Repair) are traditional measures of reliability and availability of a system \cite{koren2010fault}.

Traditional MI are centralized solutions and can easily become a single point of failure.
Although the same happens with the simple MI hardware in the blockchain-based DMS, one knows that complex systems fail more often than simpler ones.
So we can state that the $MTBF_h$ of the simple MI hardware used int the blockchain-based DMS is expected to be higher than the $MTBF_H$ of traditional MI.
Regarding the peers providing LR software execution, we claim that blockchains are based on distributed trust instead of a single point of trust.
That means the blockchain fails only if multiple stakeholders collude against the system.
Our implementation uses redundant peers (or replicas) with Byzantine consensus, which tolerates $F$ faults for $N = 3F + 1$ nodes \cite{Sousa2018}.
Every replica has its own $MTBF_R$.
The blockchain-based DMS total MTBF needs to consider that the replicas can present parallel faults, while the simple MI hardware and the set of replicas affect each other in serial faults.
In turn, availability estimative depends primarily on the MTTR \cite{koren2010fault}.
Consequently, if the organizations integrating the blockchain consensus can restore any faulty replica in an interval time  $T \leq MTBF_R *  F$, they can assure the service availability.

At the moment we write this paper, we do not have enough quantitative information for estimating the MTBF and MTTR of each solution model.
However, we can state two crucial aspects:
\begin{itemize}
 \item The blockchain-based DMS is expected to present a higher MTBF due to its simpler hardware in the field and because it does not have a single point of failure at the blockchain.
 \item The blockchain-based DMS is expected to present a better availability ratio due to its inherent fault tolerance.
\end{itemize}

Considering this discussion, we estimate the traditional MI model as presenting a fair availability level, while the blockchain-based DMS is expected to offer high availability.

\section{Challenges Ahead}
Although blockchains-based DMS is a promising approach, several challenges need to be addressed for their use.
Some of them become very clear after we proceed with our practical experiment.
We highlight the following main issues that should be addressed in future works:
%are related to the amount of data expected in some measurement applications, privacy concerns and authentication of external information providers.

\textbf{$\bullet$ The measurement Big Data:} MI usually manipulate a high amount of data. In a large-scale scenario (e.g., energy meters in a smart grid), MI can update their measurements faster, generating lots of transactions. 
A network connecting millions of meters may generate a transaction load unfeasible to be processed by existing blockchains. 
In our tests with Fabric, for instance, we reach a max throughput of 380 tps, although Androulaki et al. \cite{Androulaki2018} points out a performance more than of 2,000 tps in their benchmark. 
However, even such performance may not be enough to meet the demand for measurements on a smart grid, for instance.
In such scenarios, solutions can require different workarounds and creative alternatives.
We recall the use of endorsers, an idea that we explored with success in our experiment.
Besides, one can try to use aggregated measurements for reducing transactions in a blockchain. 
Also, one can try smarter MI for determining transactions on demand.

%Another idea is to use \emph{transaction endorsers}, a concept introduced by HyperLedger Fabric \cite{Sousa2018}. Endorsers can execute complex measuring computing, leaving only validation tasks for regular peers.

\textbf{$\bullet$ Measuring and privacy:} Measurements assigned to a specific person allow to infer information about her habits and lifestyle. 
In a blockchain with a public ledger, this problem becomes more serious. 
One needs to establish an acceptable trade-off between privacy and efficiency.
In our experiment, we faced such a problem with the vehicle license plate image and solved it by storing such information outside of the blockchain.
However, depending on the application scenario, privacy can require more sophisticated mechanisms for protecting or obfuscating identities, such as pseudonyms or identity protection layers. 
Permissioned blockchains constitute a suitable alternative once they contemplate an access control layer built into blockchain nodes \cite{Vukolic2016}. 
One can also constrain access policies in such a manner that they satisfy privacy rules and restrictions.
%Blockchains architectures also have evolved to support privacy requirements. Hyperledger 

\textbf{$\bullet$ Communication issues:} Although we consider MI as connected devices, communication can be a problem in applications demanding real-time decisions. 
That is a restriction for any DMS over asynchronous networks. 
Thus blockchain-based measuring is not appropriate for all MI applications. 
Furthermore, attacks targeting communication (e.g., DDoS) represent an additional risk, although decentralized systems such as blockchains are more resilient to such attacks than conventional cloud architectures.

\textbf{$\bullet$ Oracles authentication:} External information providers are usually called \emph{oracles} in blockchain architectures. 
In the described model, MI sensors can be seen as oracles since they are responsible for providing information from the physical world. 
Even though sensors are small components which can be protected using physical seals, sensors authentication can be necessary to assure measuring reliability.

\section{Conclusion}
In this paper we discussed how blockchains can be used to support DMS.
Due to their intrinsic security properties, blockchains can improve MI metrological assurance by imposing restrictions against potential attacks while reducing technical efforts related to regulation and control activities.
We demonstrated those properties by implementing a vehicle speed meter DMS using Hyperledger Fabric.
Our results were consistent, and they support the feasibility of our proposal. 
However, despite its promising application, blockchains pose several challenges that need to be faced. 
The main ones are related to the amount of data, privacy, communication and oracles authentication. 
Future work shall include a complete risk analysis of our blockchain-based model to develop new strategies for addressing the challenges discussed here.

% conference papers do not normally have an appendix

% use section* for acknowledgment
\section*{Acknowledgment}
%This work was partially sponsored by XXXXX XXXX XXXXX XXXX XXXXX XXXX XXXXX XXXX, grant NNN.NNN/NN-N, by XX XXX XXXX through project XX XX XXX XX.
The authors thank Perkons SA for kindly providing the speed meters dataset used in the experiments of this work.

This work was partially sponsored by the by the Coordination for the Improvement of Higher Education Personnel (CAPES), grant 99999.008512/2014-0; by the EU-BR SecureCloud project (MCTI/RNP 3rd Coordinated Call); and by FCT through the LASIGE Research Unit (UID/CEC/00408/2013) and the IRCoC project (PTDC/EEI-SCR/6970/2014).

% \subsection{The measuring Big Data}
% MI applied to massive measuring usually manipulate a big amount of data. For instance, consider a smart meter which basically integrates sensing data of voltage and electrical current sampled in high frequency to calculate energy power. In a high consume scenario, the meter updates its measurements each second, which means more than 80 thousand measurements/day. In a distributed measurement architecture using blockchains, the meter will generate a lot of transactions. When we consider a network integrating 50 million of smart meters, we have the \emph{measuring Big Data} problem and implementations using blockchains become unfeasible.
% 
% Different strategies can be explored to deal with this problem. Aggregated energy measuring already constitutes a practice used for protecting data privacy in measurements and also can be used to reduce the number of transactions in a blockchain implementation. Smarter MI can also be proposed for determining transactions on demand, when the consume exceeds an acceptable threshold. Another resource that should be investigated is the idea of \emph{transactions endorsers}, introduced by HyperLedger Fabric \cite{Vukolic2017a}. Endorsers can be used to execute the measuring complex computing, leaving to smart contracts only the task of validate the measurement.
% 
% \subsection{Measuring and privacy}
% Privacy is a constant challenge in legal metrology. Measurements assigned to a specific person allow to infer various information about her actions and habits. In a blockchain with public ledger, this problem takes on greater proportions.
% 
% The solution here is establish an acceptable trade-off among privacy and efficiency. Depending on the application scenario, privacy can require more sophisticated mechanisms for protecting or obfuscating the identity of the parties involved in the transaction. Permissioned blockchains \cite{Vukolic2017a} can constitute a suitable alternative once they contemplate an access control layer built into blockchain nodes. Access policies can be constrained such a manner they satisfies privacy rules and restrictions.
% %Blockchains architectures also have evolved to support privacy requirements. Hyperledger 
% 
% \subsection{Oracles authentication}
% External information providers are usually called oracles in blockchain architectures. In the described model, MI sensors can be seen as oracles since they are responsible for providing information from physical world. Despite sensors are small components in MI and can be protected using physical seals, sensors authentication can be necessary for assuring measuring reliability.


% An example of a floating figure using the graphicx package.
% Note that \label must occur AFTER (or within) \caption.
% For figures, \caption should occur after the \includegraphics.
% Note that IEEEtran v1.7 and later has special internal code that
% is designed to preserve the operation of \label within \caption
% even when the captionsoff option is in effect. However, because
% of issues like this, it may be the safest practice to put all your
% \label just after \caption rather than within \caption{}.
%
% Reminder: the "draftcls" or "draftclsnofoot", not "draft", class
% option should be used if it is desired that the figures are to be
% displayed while in draft mode.
%
%\begin{figure}[!t]
%\centering
%\includegraphics[width=2.5in]{myfigure}
% where an .eps filename suffix will be assumed under latex, 
% and a .pdf suffix will be assumed for pdflatex; or what has been declared
% via \DeclareGraphicsExtensions.
%\caption{Simulation results for the network.}
%\label{fig_sim}
%\end{figure}

% Note that the IEEE typically puts floats only at the top, even when this
% results in a large percentage of a column being occupied by floats.


% An example of a double column floating figure using two subfigures.
% (The subfig.sty package must be loaded for this to work.)
% The subfigure \label commands are set within each subfloat command,
% and the \label for the overall figure must come after \caption.
% \hfil is used as a separator to get equal spacing.
% Watch out that the combined width of all the subfigures on a 
% line do not exceed the text width or a line break will occur.
%
%\begin{figure*}[!t]
%\centering
%\subfloat[Case I]{\includegraphics[width=2.5in]{box}%
%\label{fig_first_case}}
%\hfil
%\subfloat[Case II]{\includegraphics[width=2.5in]{box}%
%\label{fig_second_case}}
%\caption{Simulation results for the network.}
%\label{fig_sim}
%\end{figure*}
%
% Note that often IEEE papers with subfigures do not employ subfigure
% captions (using the optional argument to \subfloat[]), but instead will
% reference/describe all of them (a), (b), etc., within the main caption.
% Be aware that for subfig.sty to generate the (a), (b), etc., subfigure
% labels, the optional argument to \subfloat must be present. If a
% subcaption is not desired, just leave its contents blank,
% e.g., \subfloat[].


% An example of a floating table. Note that, for IEEE style tables, the
% \caption command should come BEFORE the table and, given that table
% captions serve much like titles, are usually capitalized except for words
% such as a, an, and, as, at, but, by, for, in, nor, of, on, or, the, to
% and up, which are usually not capitalized unless they are the first or
% last word of the caption. Table text will default to \footnotesize as
% the IEEE normally uses this smaller font for tables.
% The \label must come after \caption as always.
%
%\begin{table}[!t]
%% increase table row spacing, adjust to taste
%\renewcommand{\arraystretch}{1.3}
% if using array.sty, it might be a good idea to tweak the value of
% \extrarowheight as needed to properly center the text within the cells
%\caption{An Example of a Table}
%\label{table_example}
%\centering
%% Some packages, such as MDW tools, offer better commands for making tables
%% than the plain LaTeX2e tabular which is used here.
%\begin{tabular}{|c||c|}
%\hline
%One & Two\\
%\hline
%Three & Four\\
%\hline
%\end{tabular}
%\end{table}


% Note that the IEEE does not put floats in the very first column
% - or typically anywhere on the first page for that matter. Also,
% in-text middle ("here") positioning is typically not used, but it
% is allowed and encouraged for Computer Society conferences (but
% not Computer Society journals). Most IEEE journals/conferences use
% top floats exclusively. 
% Note that, LaTeX2e, unlike IEEE journals/conferences, places
% footnotes above bottom floats. This can be corrected via the
% \fnbelowfloat command of the stfloats package.

% trigger a \newpage just before the given reference
% number - used to balance the columns on the last page
% adjust value as needed - may need to be readjusted if
% the document is modified later
%\IEEEtriggeratref{8}
% The "triggered" command can be changed if desired:
%\IEEEtriggercmd{\enlargethispage{-5in}}

% references section

% can use a bibliography generated by BibTeX as a .bbl file
% BibTeX documentation can be easily obtained at:
% http://mirror.ctan.org/biblio/bibtex/contrib/doc/
% The IEEEtran BibTeX style support page is at:
% http://www.michaelshell.org/tex/ieeetran/bibtex/
\bibliographystyle{IEEEtran}
% argument is your BibTeX string definitions and bibliography database(s)
\bibliography{IEEEabrv,referencias102018}
%
% <OR> manually copy in the resultant .bbl file
% set second argument of \begin to the number of references
% (used to reserve space for the reference number labels box)
% \begin{thebibliography}{1}
% 
% \bibitem{IEEEhowto:kopka}
% H.~Kopka and P.~W. Daly, \emph{A Guide to \LaTeX}, 3rd~ed.\hskip 1em plus
%   0.5em minus 0.4em\relax Harlow, England: Addison-Wesley, 1999.
% 
% \end{thebibliography}

% that's all folks

\begin{IEEEbiographynophoto}{Wilson S. Melo Jr.}
is a Researcher at the Brazilian National Institute of Metrology, Quality, and Technology (Inmetro).
He holds a Ph.D. in Computer Sciences from the Federal University of Rio de Janeiro (UFRJ).
He has more than 20 years of experience with software development and testing projects.
His main expertise regards software for industrial applications, especially solutions related to measurement, control, patterns recognizing, and cybersecurity.
More information about him can be found at https://www.researchgate.net/profile/Wilson\_Melo\_Junior.
\end{IEEEbiographynophoto}

\begin{IEEEbiographynophoto}{Alysson Bessani}
 is an Associate Professor of the Faculty of Sciences of the University of Lisboa, Portugal, and a member of LASIGE research unit. He holds a Ph.D. in Electrical Engineering from UFSC (Brazil) and was a visiting professor in Carnegie Mellow University (2010) and a visiting researcher in Microsoft Research Cambridge (2014). He is the co-author of more than 100 peer-reviewed publications on dependability, security, Byzantine fault tolerance, and cloud. More information about him can be found at http://www.di.fc.ul.pt/\texttildelow{}bessani.
\end{IEEEbiographynophoto}

\begin{IEEEbiographynophoto}{Nuno Neves}
is Professor at the Department of Computer Science, Faculty of Sciences of the University of Lisboa. He leads the Navigators research group and he is on the scientific board of the LASIGE research unit. His main research interests are in security and dependability aspects of distributed systems. Currently, he is investigator in several national and EU projects, such as SEAL and uPVN. His work has been recognized in several occasions, for example with the IBM Scientific Prize and the William C. Carter award. He is on the editorial board of the International Journal of Critical Computer-Based Systems. More information about him can be found at http://www.di.fc.ul.pt/\texttildelow{}nuno.
\end{IEEEbiographynophoto}

\begin{IEEEbiographynophoto}{Altair Olivo Santin}
received the BS degree in Computer Engineering from the PUCPR in 1992, the MSc degree from UTFPR in 1996, and the PhD degree from UFSC in 2004. 
He is a full professor of Graduate Program in Computer Science (PPGIa) and head of Security \& Privacy Lab (SecPLab) at PUCPR. 
He is a member of the IEEE, ACM, and the Brazilian Computer Society.
\end{IEEEbiographynophoto}

\begin{IEEEbiographynophoto}{Luiz F. Rust C. Carmo}
received a Ph.D. degree on Computer Science in 1994, from the LAAS/CNRS, Toulouse III – France.
Presently, he is a Senior Specialist in Computer Sciences of the Brazilian Institute of Metrology, Technology and Quality (Inmetro), General Coordinator of the Education Center. 
He is an active lecturer of both the Doctoral programs in Computer Sciences of UFRJ and in Metrology of Inmetro.
His research interests include information security, and embedded systems.
\end{IEEEbiographynophoto}

\end{document}
